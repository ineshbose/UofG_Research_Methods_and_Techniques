\documentclass[11pt,a4paper]{article}

\usepackage[a4paper,margin=2.3cm]{geometry}

\title{Users' Privacy Concerns in IoT Based Applications}
\author{}
\date{}

\begin{document}

\maketitle

\subsection*{Reference}
Psychoula, I., Singh, D., Chen, L., Chen, F., Holzinger, A. and Ning, H., 2018, October. Users' Privacy Concerns in IoT Based Applications. In 2018 IEEE SmartWorld, Ubiquitous Intelligence \& Computing, Advanced \& Trusted Computing, Scalable Computing \& Communications, Cloud \& Big Data Computing, Internet of People and Smart City Innovation (SmartWorld/SCALCOM/UIC/ATC/CBDCom/IOP/SCI) (pp. 1887-1894). IEEE.

\subsection*{Short Summary} 
This paper objective is to understand users’ perception of privacy in connection with Internet of Things (IoT) based applications: aims to figure out if and how context-sensitive factors influence users’ privacy judgment of IoT environments. In order to get a better understanding of users’ concerns about privacy, the paper used a public online survey to collect information from 236 participants all over the world and contacted 41 semi-structured interviews to better investigate different factors that could have an impact. The online survey was collecting data about users’ location, age, gender, education, familiarity with technology, and what IoT devices are owned by the participants. The outcomes of the survey and interviews are discussed in the paper, which has been observed that people are aware that there are some privacy issues, but most of the times they do not understand or know what data is being collected from them. It was also observed that elderly people are more open and willing to data sharing and are less bothered about privacy issues compared to younger generations. In summary, the core contribution of this paper is people’s perceptions, attitudes, and opinions about their data being collected and shared from their IoT devices. 

\subsection*{Why did I read this paper?}
This paper tries to understand in what areas users feel comfortable having their data collected and shared with other third-parties. It gives a summary of how an average user, depending on age or culture, would rate those areas based on how private they are to them. 

\subsection*{Personal view of the paper}
This paper makes several important contributions, breaking down how users view privacy based on their age group, ethnicity, their familiarity with technology and their educational background. It also makes a significant observation that despite most users identifying privacy risks in IoT and rating them as high privacy, they will still decide to have the IoT services if they find it useful and practical for their daily lives. The paper also provides useful graphs about users’ attitudes with data sharing, how different age groups have different views about sharing certain data, and their dating based on culture and age. It also includes tables of users’ concerns about cost, dependence of technology, trust on technology, information control and privacy \& security.

\subsection*{What problem does this paper address?}
It addresses users’ privacy concerns in their everyday life using IoT applications and tries to understand what those concerns are along with what is considered acceptable on their behalf.

\subsection*{Is it an important problem?}
Yes. A lot of users do not understand the importance of privacy and what data is collected from them on their IoT device. Some people sense that there are some privacy issues, but they do not know what those issues are or understand what data is collected of off them. Overcoming this disconnection between IoT users and their privacy could potentially bring awareness of the problems and consequences, along with stricter regulations and restrictions on users’ data collecting and data sharing.

\subsection*{What is the significance of the result and its solution?}
The results collected from this paper are highly significant and points out that regulations and security mechanisms are important factors for the acceptance of the technology. 

\subsection*{What are the claimed novel contributions of the paper?}
The paper provides contributions to privacy literature collected from a qualitative and quantitative angle to provide insights into privacy issues. 

\subsection*{What previous work is the basis for this research?}
The previous work in this paper was mostly focusing on what user characteristics to collect, such as culture and age, and various factors might affect users’ privacy. This paper has taken inspiration from other research papers that focuses users’ concerns and responses to data sharing requests based on their characterises, the importance of factors that might impact those decisions, and new risks that arise from using IoT systems.

\subsection*{What methodology has been used?}
This paper has collected quantitative data from an online survey along with qualitative data from interviews with the goal of understanding participants’ attitude towards IoT devices in addition to individuals’ opinions regarding monitoring, data sharing and privacy. 

\subsection*{Does the methodology seem appropriate for this problem?}
Online anonymous surveys along with open questions interviews seems to be the appropriate methodology as the research needs unbiased data from a large number of individuals. 

\subsection*{What conclusions are drawn from the results?}
The key conclusion is that the majority of users being aware of that there are some privacy issues despite not knowing or understanding those issues, many users still decide to have the offered IoT services if its useful and practical for their daily lives. It also shows that older people are more acceptant to data sharing while being less concerned about privacy issues compared to younger generations. 

\subsection*{Are they valid?}
Yes. The paper shows the different perspective users have about their privacy and data being shared based on their age and culture and makes reasonable arguments about why they have different opinions. 

\subsection*{What did I learn?}
From this paper I learned that not all people have the same beliefs and opinions about privacy and data sharing, and potentially why there are different perspectives on privacy matters.

\subsection*{What (if anything) would I have done differently?}
The paper tried to cover many factors to why individuals’ opinions might differ about privacy and one thing that I might have done differently is to go into more in-depth about individuals’ daily life to see how they are dealing with privacy issues as nearly everything nowadays asks to collect data from their users, and if they are deciding to not use something due to those privacy issues.

\end{document}