\documentclass[11pt,a4paper]{article}

\usepackage[a4paper,margin=2.3cm]{geometry}

\title{A Comprehensive Study of Security of Internet-of-Things}
\author{}
\date{}

\begin{document}

\maketitle

\subsection*{Reference}
Mosenia, A., \& Jha, N. K. (2016). A comprehensive study of security of internet-of-things. IEEE Transactions on emerging topics in computing, 5(4), 586-602.

\subsection*{Short Summary} 
Mosenia \& Jha attempted to address and summarise threats to IoT, especially at the time of its emergence, over five sections. The paper relies on the three-level reference model that is introduced in Section II along with two other models (Figure 1); the important levels out of these are Edge devices (mostly RFID tags are used as an example), Communication and Edge computing nodes (like an RFID reader – different from a tag!). Further in the paper, since there are examples provided where a threat could put things at risk, Section III aims to help the reader understand IoT applications by listing Smart vehicles, buildings, health monitoring, and more, and then discussing the motivations for attacks and the need for a thing to be secure by meeting requirements in the Table 1 of the paper. From Section IV to VI, the paper is technical, however, includes definitions and examples, discussing vulnerabilities, countermeasures, and emerging challenges respectively. A lot of sections and explanations are from Figure 3. While there were no experiments held by the authors for this paper to discuss vulnerabilities, other papers where security tests were conducted have been referred and cited.

\subsection*{Why did I read this paper?}
Given the title of the paper, it seemed to be a good introduction to start the paper summaries for the RMT Coursework. 

\subsection*{Personal view of the paper}
I expected this paper to be a high-level overview, and while it was that, there were technical bits as well – like Section IV, but despite that, all of it was easy to read. However, I felt that Section V was slightly over the place and not in perfect order as it was mostly referring to Figure 3 – which, again, was a very useful illustration. 

\subsection*{What problem does this paper address?}
This paper addresses the side effect of the emergence of IoT where potential threats and attacks against the security or privacy of a person has also grown drastically. 

\subsection*{Is it an important problem?}
Yes. This problem has global awareness; however, I believe papers in this field create a bigger impact to make people reflect deeper. 

\subsection*{What is the significance of the result and its solution?}
The ideas discussed by the authors isn’t highly significant as it is nothing new, however, it is still important and useful since countermeasures are discussed in the three-layer model. 

\subsection*{What are the claimed novel contributions of the paper?}
The vulnerabilities and countermeasures of IoT laid out in a three-level model. 

\subsection*{What previous work is the basis for this research?}
The security threats had been studied prior to this. More importantly, the reference models that the paper relies on were also shared through different papers. 

\subsection*{What methodology has been used?}
The paper does not use any methodology of its own but refers to other papers to explain vulnerabilities. 

\subsection*{Does the methodology seem appropriate for this problem?}
The choice of methods was appropriate as each example for the reader to understand more. 

\subsection*{Has the methodology been performed correctly?}
The aim of the methods was to showcase vulnerabilities and countermeasures which were successful given the scenario. 

\subsection*{What conclusions are drawn from the results?}
The paper acknowledges that there would be threats that are not yet known (for example Edge computing level), but there are also ones that are not recognised by everyone in the domain of IoT and should be addressed by researchers and thing manufacturers. 

\subsection*{Are they valid?}
The paper is correct to conclude that the threats should be addressed aggressively as the motivations for threats and attacks could be an individual’s privacy. 

\subsection*{What did I learn?}
I learnt about threats to IoT through the three-level model, which was very helpful. 

\subsection*{What (if anything) would I have done differently?}
The flow of the paper was slightly difficult for me, going from vulnerabilities of each layer to countermeasures. I believe that having both vulnerabilities and countermeasures in the same section for each layer would be easier. 

\end{document}