\documentclass[11pt,a4paper]{article}

\usepackage[a4paper,margin=2.3cm]{geometry}

\title{Study to improve security for IoT smart device controller: drawbacks and countermeasures}
\author{}
\date{}

\begin{document}

\maketitle

\subsection*{Reference}
Su, X., Wang, Z., Liu, X., Choi, C. and Choi, D., 2018. Study to improve security for IoT smart device controller: drawbacks and countermeasures. Security and Communication Networks, 2018. 

\subsection*{Short Summary} 
This paper talks about how different categories of IoT devices such as smart and mobile devices are connected automatically using a device controller, such as a smartphone, and why the controller must be more secure compared to traditional security mechanism. It goes over different current security mechanisms a user can protect their smartphone, using a text password, a graphical password, a PIN, a signature or a fingerprint. The paper investigates and categorizes typical attacks used on the security mechanisms and their emerging issues which then analyses standard and existing countermeasures, with proposing several new representative methods for countermeasures. 

\subsection*{Why did I read this paper?}
This paper talks about several ways an attacker can gain knowledge of their victims’ password, factors that make it easy for an attacker to gain that knowledge. It then suggests countermeasures for those existing attacks based on how easy it makes it for an attacker but also taking into consideration users’ usability of the device.

\subsection*{Personal view of the paper}
This paper makes several important investigations, mostly on how a typical user can be exposing their password without knowing about it. It goes into detail about each way a user can protect their smartphone and then explains several attacks an attacker can use to gain your password on each method. It also contains a table of the security comparison between the attacking techniques against the existing security mechanisms and proposed schemes of securing your phone, another table which takes user’s usability into account, and a final table that includes the advantages and disadvantages of the proposed schemes. I think this is a good paper because of the variety of ways a user can protect themselves and the different attacking techniques an attacker can use, however the paper lacks to mention the size of their experiment as it only includes “that participants gave positive reactions”.

\subsection*{What problem does this paper address?}
This paper focuses on the current security threads attacking mobile device structure defects and human errors and calls attention to the vulnerabilities in the existing security mechanisms.  

\subsection*{Is it an important problem?}
It is not such an important problem since people tend to carry their mobile devices with them at most times, however in a scenario where you lose your phone then it becomes more important. Despite not being such as important problem, it is important to be aware of how good your security mechanism is against an attacker and be cautious about the attacks that can used against you. 

\subsection*{What is the significance of the result and its solution?}
The ideas in this are highly significant and proposes new security schemes that can be used to extend even further the protection of your mobile devices. 

\subsection*{What are the claimed novel contributions of the paper?}
New security schemes that overcome several issues with standard security mechanisms. 

\subsection*{What previous work is the basis for this research?}
This paper analyses security threats and categorizing them along with their countermeasures. It goes into details about each security thread, what an attacker is looking for and how they execute it, along with how difficult it would be for them to execute those attacks. It also provides demonstrational figures to give the user a better understanding of what to look after in order to protect themselves. 

\subsection*{What methodology has been used?}
No experiments in the paper. However, the paper describes several techniques an attacker can use to gain knowledge of a users’ credentials along with demonstrational figures showing the attack in action. It also describes new proposed schemes that can prevent these attacks along with providing demonstrational figures about the new schemes. Also, the existing and proposed schemes were simulated and tested using MIT App Inventor 2. 

\subsection*{Does the methodology seem appropriate for this problem?}
The methodology used seems appropriate as it takes into account nearly all possible security mechanisms along with their attacks and gives countermeasures for all of them while suggesting new security mechanisms. 

\subsection*{What conclusions are drawn from the results?}
The conclusion from this paper is that despite the new proposed schemes are very secure against emerging attacks compared to existing schemes, some of the schemes suggested have insufficient usability and not all users would be willing to use the new schemes. 

\subsection*{Are they valid?}
They are somewhat valid. The paper includes advantages and disadvantages of the new proposed schemes, and it also mentions that despite low input speed and memory problem, participants gave positive reactions. However, it does not mention the number of participants, whether it was the people that created the simulations that participated or people unfamiliar with those techniques. 

\subsection*{What did I learn?}
I was already familiar with the different security schemes and with the majority of their vulnerabilities, however I was not aware of the new proposed schemes and how they can make mobile devices more secure and harder to attack. 
\subsection*{What (if anything) would I have done differently?}
As this paper aims to find countermeasures to prevent the present vulnerabilities, I would have hired/contacted ethical hackers since it’s their job to know all these vulnerabilities and have them try to find faults in the new proposed schemes as they are more highly knowledgeable in that area and can have a different perspective that the writers.

\end{document}