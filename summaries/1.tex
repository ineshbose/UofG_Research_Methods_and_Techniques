\documentclass[11pt,a4paper]{article}

\usepackage[a4paper,margin=2.3cm]{geometry}

\title{A Survey of IoT security threats and defences}
\author{}
\date{}

\begin{document}

\maketitle

\subsection*{Reference}
Ahmed, H.I., Nasr, A.A., Abdel-Mageid, S. and Aslan, H.K., 2019. A survey of IoT security threats and defenses. International Journal of Advanced Computer Research, 9(45), pp.325-350.

\subsection*{Short Summary} 
This paper describes the security problems in regard to IoT devices and proposes several countermeasures against the problems researchers come across in this field.  It outlines the architecture of IoT devices, how they are split into three different layers, Application layer, Network layer, and Sensing layer. For each layer it provides a detailed description of the attacks that can be used against it and what kind of attacks are currently used in the wild. Along with the descriptions, it provides detection methods for the attacks that can be detected so far, along with countermeasures for all of the attacks on each individual layer. The paper then categorises the challenges researchers face in IoT security which security agents must prove the reliability, economy, efficiency and effectiveness of the security. It ends the paper with a description of the evaluation strategies needed for the security of IoT infrastructure.  

\subsection*{Why did I read this paper?}
This paper gives an extensive list of the currently present attacks used against IoT devices and provides detection methods for the majority of them along with countermeasures for all of the attacks mentioned. 

\subsection*{Personal view of the paper}
This paper makes a solid summary of the most known and frequent attacks used against IoT devices and provides detection solutions to capture those attacks and prevent them from happening. It also provides an extensive list of countermeasures for all of the attacks so users or manufacturers can take into account when implementing/upgrading their IoT devices. In addition to the attacks, the paper provides several challenges security agents face when they must prove the reliability, economy, efficiency and effectiveness of the IoT device’s security. 

\subsection*{What problem does this paper address?}
It covers the most recurring attacks on all of IoT architecture layers and mentions the challenges it may face along with their effects on IoT security. 

\subsection*{Is it an important problem?}
Yes. With the number of IoT devices connected to the internet dramatically increasing, there will be even more vulnerable devices out in the wild for attackers to infect. While the current security problems in IoT devices are being ignored, with the number of devices increasing, the chances of new vulnerabilities to be created as high. Overcoming these security problems could allow IoT devices to become secure while protecting users’ data and potentially opening a new path in the development of IoT. 

\subsection*{What is the significance of the result and its solution?}
The ideas in this are highly significant and could finally bring awareness to users and manufacturers in hope for more secure IoT devices. 

\subsection*{What are the claimed novel contributions of the paper?}
An increase in the security of IoT devices and an overall more secure internet. 

\subsection*{What previous work is the basis for this research?}
This paper has taken inspiration from several other research paper, by explaining the architecture of the IoT devices, giving a detailed description about attacks that can be used against each layer and proposing countermeasures for those attacks along with the challenges that need to be taken into consideration. With these research topics combined, this paper provides a comprehensive overview of IoT devices and its faults. 

\subsection*{What methodology has been used?}
There were no experiments in this paper. However, it breaks down the architecture layers of IoT devices and specifies attacks that can be used against each layer. It proceeds by descripting the attacks and provides countermeasures for those attacks along with a few detection methods. 

\subsection*{Does the methodology seem appropriate for this problem?}
Breaking down the architecture layers of IoT devices and describing the attacks that can be used against each layer is fine for this paper as it gives a good overview of what those attacks are. 

\subsection*{What conclusions are drawn from the results?}
The conclusions are that there are several vulnerabilities in IoT devices that need to be taken into consideration but while the number of IoT devices dramatically increasing is becoming extremely difficult to overcome them. 

\subsection*{Are they valid?}
Yes. The paper describes broadly the attacks that are currently known while providing countermeasures. 

\subsection*{What did I learn?}
I was familiar with IoT devices being vulnerable, but this paper has provided a tremendous list of attacks that I was not aware of. It was also interesting to learn the challenges that need to be considered and their effects on IoT security. 

\subsection*{What (if anything) would I have done differently?}
I would have liked to see some evidence on those attacks mentioned and maybe a collection of the most frequent attacks used.

\end{document}